
\documentclass{simplecv}
\usepackage[utf8]{inputenc}
\usepackage[margin=2cm]{geometry}
\usepackage[english,russian]{babel}





\begin{document}

\name{Ashurov Sindorjon Ahmadjon o`g`li}
\personalinfo{Parkent tumani Xisorak qishlog`i} {Istiqbol ko`chasi, 70-uy}{111309}
             {ashurovsindor@gmail.com}{99 841 6097}


\cvsection{Profil}
\cvitem{Yadro fizikasi va yadro texnologiyalari yo`nalishi bo`yicha magistr}
\cvitem{Python dasturlash tili bo`yicha mutaxassis}
\cvitem{Nanomeditsina sohasida maxsus kurs bitiruvchisi}



\cvsection{Ta'lim}
\cvevent{2015--2019}{Fizika fani bo`yicha bakalavr}
        {O`zbekiston Milliy Universiteti}{Toshkent}


\cvevent{2018--2019}{Nanomeditsina maxsus kuri}
        {by prof R.Letfulin from Rose Hulman University}{Toshkent}

\cvevent{2019--2021}{Yadro fizikasi va yadro texnologiyalari yo`nalishida magistr}{O`zbekiston Milliy Universiteti}{Toshkent}



\cvsection{Ish}

\cvevent{2018/09--2019/01}{Informatika fani o`qituvchisi}
        {320-maktab}{Toshkent}

\cvevent{08/2019--12/2020}{8-darajali injener}{"Fizika-Quyosh" IIChB}{Parkent}
\cvevent{2019/09--hozirgacha}{Fizika fani o`qituvchisi}{42-maktab}{Toshkent}




\cvsection{IT }

\cvskill{C++}{7}
\cvskill{Python}{9}
\cvskill{MATLAB}{7}
\cvskill{FORTRAN}{8}
\cvskill{Git}{6}
\cvskill{Latex}{10}



\cvsection{Maqolalar}

\cvitem{
\textbf{Ashurov S}, 
\newblock O`zbekistondagi Katta Quyosh Pechining ekspluatatsion parametrlarini qisman quyosh tutilishi vaqtida o`rganish

}

\cvitem{
\textbf{Ashurov S}, 
\newblock Rux oksidi asosidagi qattiq qotishmalarda fotolyuminessensiya hodisalari

}

\cvitem{\textbf{Ashurov S},
\newblock Spinli elektronika va uning elementlari}

\cvsection{Chet tillari}

\cvlangitem{O`zbek (ona tili)}
{English (fluent)}
\cvlangitem{Русский (родной)}
{Fran\c ais (interm\'ediaire)}



\end{document}
